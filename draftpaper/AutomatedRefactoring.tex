\documentclass[conference,compsoc,a4paper]{IEEEtran}

\usepackage{float}
\usepackage[utf8]{inputenc}
\usepackage[final]{pdfpages}
\usepackage[cmex10]{amsmath}
\usepackage{amssymb}
\usepackage{comment}
\usepackage{setspace}
\usepackage{caption}
\usepackage{subcaption}
\usepackage{acronym}
\usepackage{url}

\usepackage{multirow}
\usepackage{pbox}
\usepackage{xspace}
\usepackage{balance}

\newcommand{\lk}[1]{\textcolor{orange}{\textbf{[LK] #1}}}
\newcommand{\pf}[1]{\textcolor{green}{\textbf{[PF] #1}}}


\begin{document}

\title{Automated Refactoring}
\author{
  \IEEEauthorblockN{Lisa Maria Kritzinger}
  \IEEEauthorblockA{Johannes Kepler University Linz\\
    1255353\\
    Email: kritzinger@gmx.net}
  \and
  \IEEEauthorblockN{Peter Feichtinger}
  \IEEEauthorblockA{Johannes Kepler University Linz\\
    1056451\\
    Email: shippo@gmx.at}
}

% make the title area
\maketitle


\begin{abstract}
Refactoring, the restructuring of a software system without changing its semantics, is essential in software evolution. 
Manual refactoring can be time-consuming and error-prone, so tool support is desirable when making large changes. In 
this article, we will explore a number of publications on automating different refactoring tasks, from just making code 
more compact to introducing objects into a C codebase.

\pf{Should the abstract be longer?}
\end{abstract}

\begin{IEEEkeywords}
Software restructuring, automatic refactoring, tool support, software evolution.
\end{IEEEkeywords}

% For peer review papers, you can put extra information on the cover
% page as needed:
% \ifCLASSOPTIONpeerreview
% \begin{center} \bfseries EDICS Category: 3-BBND \end{center}
% \fi
%
% For peerreview papers, this IEEEtran command inserts a page break and
% creates the second title. It will be ignored for other modes.
\IEEEpeerreviewmaketitle


\section{Introduction}

Refactoring is the process of restructuring a software system without changing its semantics. It is used to increase 
readability and maintainability of software, reduce its complexity, or change the architecture of a system. Refactoring 
is essential in software evolution, because as a system is adapted to new requirements it inevitably becomes more 
complex and drifts away from its original design. This makes maintenance more difficult and reduces the software 
quality. Refactoring can then help to bring the system back to its original design and into a more maintainable state, 
improving code quality in the process.

However, manual refactoring without any tool support can be error-prone and time-consuming. There are various tools 
available for supporting elementary refactorings like renaming a variable or introducing an additional parameter to a 
function, often built into the used \emph{Integrated Development Environment} (IDE) itself. But even with tool support, 
manual refactoring can still be too complicated or just tedious, and tool support for automating refactoring tasks is 
desirable in a number of cases.

In this article we're going to highlight some recent and not-so-recent contributions in the field of automatic 
refactoring, ranging from simple tasks like making code more compact \cite{sparta}, to more complicated tasks like 
introducing object-orientation into a C codebase \cite{cpp}.


\section{Background}

Text


\section{Conclusion}

Text


\section*{Acknowledgments}

Text


\bibliographystyle{IEEEtran}
\bibliography{AutomatedRefactoring}

% TODO Place appropriately
%\balance

% that's all folks
\end{document}
