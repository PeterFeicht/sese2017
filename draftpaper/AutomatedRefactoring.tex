\documentclass[conference,compsoc,a4paper]{IEEEtran}

\usepackage[utf8]{inputenc}
\usepackage[american]{babel}
\usepackage[babel]{csquotes}

\usepackage[tableposition=top]{caption}

\usepackage[usenames,dvipsnames]{xcolor}
\usepackage{graphicx}

\usepackage[final]{pdfpages}
\usepackage[cmex10]{amsmath}
\usepackage{amssymb}
\usepackage{acronym}

\usepackage{xspace}
\usepackage{balance}

\usepackage{listings}
\lstset{
  basicstyle=\ttfamily\footnotesize,
  stringstyle=\color{Plum}\slshape,
  commentstyle=\color{OliveGreen},
  keywordstyle=\color{Plum},
  showstringspaces=false,
  tabsize=2,
  frame=single
}

\usepackage[
  pdflang=en,
  colorlinks=true,
  unicode=true,
  pdfstartview=,
  allcolors=black
]{hyperref}
\hypersetup{
  pdftitle={Refactoring Techniques and Automated Approaches Through Tool Support},
  pdfauthor={Lisa Maria Kritzinger, Peter Feichtinger}
}

\newcommand{\code}[1]{{\small\ttfamily #1}}

\newcommand{\lk}[1]{\textcolor{orange}{[LK] #1}}
\newcommand{\pf}[1]{\textcolor{green}{[PF] #1}}

\newcommand{\JDEvAn}{\mbox{JDEvAn}\xspace}


\begin{document}

\title{Refactoring Techniques and Automated Approaches Through Tool Support}
\author{
  \IEEEauthorblockN{Lisa Maria Kritzinger}
  \IEEEauthorblockA{Johannes Kepler University Linz\\
    1255353\\
    Email: \href{mailto:kritzinger@gmx.net}{kritzinger@gmx.net}}
  \and
  \IEEEauthorblockN{Peter Feichtinger}
  \IEEEauthorblockA{Johannes Kepler University Linz\\
    1056451\\
    Email: \href{mailto:shippo@gmx.at}{shippo@gmx.at}}
}

% make the title area
\maketitle


\begin{abstract}
Refactoring, the restructuring of a software system without changing its semantics, is essential in software evolution. 
The notion of refactoring has been embraced by many object oriented software developers as a way to accommodate
changing requirements. If applied well, refactoring improves the maintainability of software, but manual refactoring 
can be time-consuming and error-prone, so tool support is desirable when making large changes. In this article, we will 
explore a number of publications on automating different refactoring tasks, from just making code more compact to 
introducing objects into a C codebase.

Additionally, we will provide comparisons which will show the advantages and disadvantages of different approaches, 
regarding the performance of the software after refactoring, the applicability of a certain approach, or the simplicity 
of the application of an approach.
\end{abstract}

\begin{IEEEkeywords}
Software restructuring, automatic refactoring, tool support, software evolution.
\end{IEEEkeywords}

% For peer review papers, you can put extra information on the cover
% page as needed:
% \ifCLASSOPTIONpeerreview
% \begin{center} \bfseries EDICS Category: 3-BBND \end{center}
% \fi
%
% For peerreview papers, this IEEEtran command inserts a page break and
% creates the second title. It will be ignored for other modes.
\IEEEpeerreviewmaketitle


\section{Introduction}

Refactoring is the process of restructuring a software system without changing its semantics. It is used to increase 
readability and maintainability of software, reduce its complexity, or change the architecture of a system. Refactoring 
is essential in software evolution, because as a system is adapted to new requirements it inevitably becomes more 
complex and drifts away from its original design. This makes maintenance more difficult and reduces the software 
quality. Refactoring can then help to bring the system back to its original design and into a more maintainable state, 
improving code quality in the process.

However, manual refactoring without any tool support can be error-prone and time-consuming. There are various tools 
available for supporting elementary refactorings like renaming a variable or introducing an additional parameter to a 
function, often built into the used \emph{Integrated Development Environment} (IDE) itself. But even with tool support, 
manual refactoring can still be too complicated or just tedious, and tool support for automating refactoring tasks is 
desirable in a number of cases.

In this article we're going to highlight some recent and not-so-recent contributions in the field of automatic 
refactoring, ranging from simple tasks like making code more compact \cite{sparta}, to more complicated tasks like 
introducing object-orientation into a C codebase \cite{cpp}.
%We will also investigate the benefit from refactoring a software through replacing conditionals by using polymorphism 
%\cite{polymorphism}, as well as the approach of using design differencing for automated refactoring \cite{design-diff}.


\section{Automated Refactoring Approaches}

There are many refactoring tools available, most of which focus on performing specific refactorings as requested by the
developer, other tools are based on search-based refactoring. Although there are already many tools for software 
refactoring, there are still techniques without tool support. The following section will give an overview of some 
papers which describe some techniques and tools by summarizing the main points.

\subsection{Restructuring Legacy C Code into C++} \label{sec:cpp}

This is an older paper from 1999 by Richard Fanta and Václav Rajlich of \emph{Wayne State University} in Detroit, MI, 
USA~\cite{cpp}. They did a case study on the Mosaic browser, an early web browser implemented in C. Their approach uses 
a number of discrete refactorings. By combining those, a C \code{struct} or a number of related variables can be 
transformed into a C++ class, with related functions becoming member functions of that class.

The following two sections will briefly explain the implemented refactorings and their use in the whole restructuring 
process, respectively.

\subsubsection{Refactoring Tools}

This section briefly summarizes the implemented tools used in the restructuring process. Each tool has specific 
restrictions placed on when it can be applied, which simplifies the tool and makes sure the code is in a consistent 
state after its application.

\begin{enumerate}
  \item The \textbf{variable insertion} tool inserts a selected variable into a class as a static or non-static member. 
  The programmer needs to specify the variable which should be inserted, as well as the class---for a static 
  variable---or the instance---for a non-static variable---it should be inserted into.
  
  \item Another tool \textbf{makes access to a non-local variable explicit} by introducing an explicit parameter for 
  the accessed variable, redirecting all accesses inside the function to that parameter, and finally adding an actual 
  parameter for the variable at every call of the function. The same is also possible in reverse to \textbf{make access 
  implicit}.
  
  \item To \textbf{add a parameter} to a function, another tool is used that adds the parameter to the formal parameter 
  list. After selecting the instance to be passed as the new parameter for each call, the tool inserts that instance as 
  a parameter to the call.
  
  \item Finally there is a tool for \textbf{changing the access specifier of a class member}, which just checks that a 
  certain change doesn't make the code inconsistent.
\end{enumerate}

\subsubsection{Restructuring Scenario}

The refactoring tools described in the previous section are used at various steps in the whole process. The complete 
restructuring process thus involves both actions by a human as well as use of the tools, and is divided into three 
phases.

\begin{enumerate}
  \item Data-only classes are created from a number of variables, if necessary. In case there is already a C 
  \code{struct}, this step can be skipped.
  
  \item After creating the desired classes, possible clones are removed. Clones may result from the same domain concept 
  being implemented at various places in the code.
  
  \item Lastly, the user specifies functions which should be added to a class as a member function. These may be 
  functions that have the target class as a parameter, access it through a global variable, or have individual members 
  of the target class as parameters.
\end{enumerate}

\subsubsection{Results}

For their evaluation, Fanta and Rajlich selected a subsystem of 3000 lines from the Mosaic codebase. Of these 3000 
lines they were able to encapsulate about 60\% of the code into 12 classes. As an example where clone removal was 
necessary the give the URL class, which was extracted separately at five different locations.

\subsection{Performance Impact of Polymorphism} \label{sec:polymorphism}

Programmers often argue that they cannot afford refactoring their code to use polymorphism instead of large 
conditionals, because of a negative impact on performance. Serge Demeyer investigated this claim in 2002 
\cite{polymorphism} by comparing the performance of a program using large conditionals against the same program with 
conditionals replaced by polymorphism. The comparison showed that C++ programs modified in this way usually perform 
better, or at least as good in the case of a small number of types.

The approach of introducing polymorphism to replace large conditional chains (or large \code{switch} statements) is of 
course not always applicable. But when the same conditional logic appears in different parts of a program, 
maintainability will likely suffer. With duplicated logic it is easy for a programmer to modify different parts of the 
program in incompatible ways, or to just forget to evolve one instance in case another is modified.

Demeyer describes three variants of complex conditional logic, and ways to remove them.

\begin{enumerate}
  \item \textbf{Client Type Checks}
  If a client is testing the type of a certain provider object, it can be refactored by moving code from the client to 
  the provider. The special case that a client tests whether a provider is \code{null} or empty can be refactored by 
  introducing a special \emph{null object} \cite{nullobject}.
  
  \item \textbf{Self Type Checks}
  The case that an object is testing an attribute serving as some kind of type-tag can be refactored by creating a new 
  subclass for each leg of the conditional and moving the code as a polymorphic method into the subclass. Listings 
  \ref{lst:conditional} and \ref{lst:polymorphic} show an example of this case. A single class containing large methods 
  with conditional logic in \autoref{lst:conditional} is replaced by several classes in \autoref{lst:polymorphic}, with 
  each of the classes implementing one leg in the conditional chain.
  
  If an object changes its state dynamically, it can be refactored by introducing a state object   
  \cite[pp.~305--313]{designpatterns}.
  
  \item \textbf{Transforming Conditionals into Registration}
  Another case are clients which test the type of a series of objects before performing a certain action. This can be 
  refactored by a central registration mechanism, which acts as a mediator between objects providing services and 
  clients requesting services.
\end{enumerate}

\subsubsection{Results}

Demeyer concludes that for all but the very simplest conditionals, replacing them with polymorphism has no negative 
impact on performance. In fact it has an increasingly positive impact for larger conditionals, and mostly no impact 
when replacing \code{switch} statements.

Although this method is easy to use, there is no tool support to automate the process. Therefore it is difficult to 
motivate programmers to make use of this method.

\lstinputlisting[language={[11]C++},float=t,label=lst:conditional,
  caption={A class with type tag and large conditionals, showing both \code{if else} and \code{switch} statements}]
  {code/ConditionalWidget.cpp}

\lstinputlisting[language={[11]C++},morekeywords={override},float=t,label=lst:polymorphic,
  caption={Class from \autoref{lst:conditional} replaced by several classes and polymorphic methods}]
  {code/PolymorphicWidget.cpp}

\subsection{Design Differencing} \label{sec:design-diff}

In their paper from 2012, Moghadam and Ó Cinnéide \cite{design-diff} propose a novel approach for refactoring a 
software system based on a desired design and information extracted from the source code.

The process for this approach is illustrated in \autoref{fig:designdiff}. The programmer creates a desired design for 
the system. The desired design is based on the current design of the system, as well as an understanding of how the 
system may be required to evolve in the future. The software system is then refactored based on the difference of the 
desired design and the current design, which is extracted from the source code.

The whole process is divided into two phases. In the \emph{detection phase}, information on the current design is 
extracted from the source code, which can then be used to create the desired design. After the desired design is 
created, necessary refactorings are calculated based on the difference of actual and desired design. In the following 
\emph{reification phase}, previously calculated refactorings are applied to the system in order to change its design.

\begin{figure}[t]
  \centering
    \includegraphics[width=\linewidth]{figures/designdiff.png}
  \caption{Refactoring process using design differencing \cite{design-diff}}
  \label{fig:designdiff}
\end{figure}

This refactoring approach is supported by two tools in particular: \JDEvAn (Java Design Evolution and Analysis) 
\cite{JDEvAn}, which is used for fact extraction and design differencing, and Code-Imp (Combinatorial Optimization for 
Design Improvement) \cite{DBLP:journals/jss/OKeeffeC08, DBLP:conf/icse/MoghadamC11}, which is used for applying 
refactorings to the software system. The following sections will give a brief description of both tools.

\subsubsection{JDEvAn}

\JDEvAn is developed at the University of Alberta. It is an Eclipse plugin which analyzes a software system's 
design-evolution history and provides information about the system's history. The plugin contains a Java fact 
extractor, a query-based change-pattern detection module, and a design differencing algorithm. The Java fact extractor 
extracts a logical UML design model from Java source code. The design differencing algorithm uses lexical and 
structural similarity to automatically recover differences between one version of a system and the next \cite{Xing2007}.
\JDEvAn provides a refactoring detection module which categorizes detected differences as refactoring instances 
\cite{DBLP:conf/wcre/XingS06}.

The process initially extracts two UML models from the source code corresponding to two versions of the software 
system. Afterwards, the two models are compared and the differences between them are detected. Finally the detected 
differences are categorized as design-level refactoring instances. For example, a method which is moved from a class to 
a subclass in the desired design is detected as a number of \emph{move method} differences.

\subsubsection{Code-Imp}

Code-Imp is developed by O'Keeffe, Ó Cinnéide, and Moghadam \cite{DBLP:journals/jss/OKeeffeC08, 
DBLP:conf/icse/MoghadamC11} as a fully automated refactoring framework for improving the design of existing programs. 
It takes Java source code as input and provides a refactored version of the program as output.

The refactoring process of Code-Imp is driven by a search technique, steepest-ascent hill climbing. With this 
technique, the next refactoring to be applied is the one that produces the best improvement in the so called fitness 
function, a function which measures how good the program is.

\subsubsection{Results}

The benefit of refactoring software using tools like \JDEvAn and Code-Imp is that it enables automated refactoring 
towards a high-quality desired design, and hence improves maintenance productivity. Moghadam and Ó Cinnéide showed the 
efficacy of this approach, their findings were that the original program could be refactored to the desired design with 
an accuracy of over 90\%, hence demonstrating the viability of automated refactoring using design differencing.

\subsection{The Spartanizer} \label{sec:sparta}

This is a recent paper by Yossi Gil and Matteo Orrù, which was presented at this year's SANER. It describes their tool 
called \emph{The Spartanizer} \cite{sparta}, which is an Eclipse plugin for automatic refactoring of Java code. The 
tool is still actively being developed on GitHub\footnote{\url{https://github.com/SpartanRefactoring/Main}} and has 
quite a number of contributors.

The supported refactorings are implemented in classes inheriting from the abstract \code{Tipper} class. A tipper 
represents a rewrite rule that can be applied to a specific type of AST node. Multiple tippers can be applied in 
succession to the whole project. An example is given in listings \ref{lst:c0} and \ref{lst:c1}. \autoref{lst:c0} shows 
a generic class with \code{equals} and \code{hashCode} methods generated by Eclipse, which are quite verbose for a 
class with only one member, because they are aimed at the general case of many members. The \emph{spartanized} version 
of the class is shown in \autoref{lst:c1}, it is much more compact than the first version and has done away with much 
of the verbosity in the generated methods.

\lstinputlisting[language=Java,float=tb,label=lst:c0,caption={A generic class with methods generated by Eclipse}]
  {code/C0.java}

\lstinputlisting[language=Java,float=tb,label=lst:c1,caption={A spartanized version of the class in \autoref{lst:c0}}]
  {code/C1.java}

After each modification to a source file, it is parsed and available suggestions are displayed as info markers on the 
source code. From there, it is possible to apply a refactoring to a selected scope (method only, the whole file, etc).

\subsubsection{Results}

This paper is unique in our list, because it describes a tool that is actively maintained and can be used in 
production. The tool is available from the Eclipse Marketplace\footnotemark{} with over a hundred installs each month.

\footnotetext{\url{https://marketplace.eclipse.org/content/spartan-refactoring-0}}


\section{Automated Refactoring in General}

As we have already seen from just a few papers, the spectrum of automated refactoring is quite wide. Different 
techniques require more or less interaction from the user, and their scope ranges from localized changes (such as 
introducing polymorphism into a class hierarchy as seen in \autoref{sec:polymorphism}) to changes to the whole software 
system (such as changing the design of the whole system as seen in \autoref{sec:design-diff}).

There are also different motivations for using automated refactoring, or refactoring in general. One might just want to 
make code more readable on the one hand, on the other hand the motivation might be that a system requires profound 
changes in order to make it more maintainable.
Mens et al.\ identify three steps in the refactoring process \cite{DBLP:conf/iwpse/MensTM03} which are also illustrated 
in \autoref{fig:process}:
\begin{enumerate}
  \item detect when an application should be refactored;
  \item identify which refactorings should be applied and where; and
  \item perform the refactorings.
\end{enumerate}

\begin{figure}[htb]
  \centering
    \includegraphics[width=0.8\linewidth]{figures/process}
  \caption{The refactoring process according to \cite{DBLP:conf/iwpse/MensTM03}}
  \label{fig:process}
\end{figure}

They also note that of those steps, only the third one is currently well supported by tools. This is still true today, 
more than 13 years later. We saw that some tools are able to identify where refactorings can be applied, as in the case 
of the \emph{Spartanizer}, while others need to be guided by the developer.

Determining when a software system has a need for refactoring is supported by tools for the detection of \emph{code 
smells}, symptoms of poor design and implementation choices \cite{DBLP:books/daglib/0019908}. Various such tools are 
already available. Palomba et al.\ use, for example, information from the change history of a software system to detect 
smells \cite{DBLP:conf/kbse/PalombaBPOLP13}.
However, we think that combining this with the second point from above, actually identifying how a smell can be fixed 
by applying certain refactorings at specific parts in the code, is essential for the success of those tools.


\section{Conclusion}

We showed various different approaches to automatic refactoring.

Fanta and Rajlich \cite{cpp} use different refactoring tools in succession, guided by the user, to introduce classes 
into a C codebase. They require the user to specify which variables should be grouped into a C++ class and to specify 
which functions should be moved into that class.

Demeyer \cite{polymorphism} showed that in many cases, replacing large conditionals by polymorphism to dispatch between 
different behaviors doesn't necessarily have an impact on performance and, if it does, then usually a positive one.

Moghadam and Ó Cinnéide \cite{design-diff} show how a software system can be automatically refactored based on a 
desired design and information extracted from the code. They use two different tools, one to extract the actual design 
from the software and generate necessary refactorings by comparing it to a desired design, the other to actually apply 
the generated refactorings to the software system.

Gil and Orrù \cite{sparta} present a tool that can be used to automatically apply a large number of refactorings to 
(parts of) a system, with the goal to make its code more compact. The tool is integrated into the Eclipse IDE and 
provides opportunities for the user to apply one  or more refactorings to different parts of the codebase.

In summary, we can say that a large number of automated tools are already available. These tools support developers in 
refactoring software, which is particularly important for the common problem of extending an existing program with new 
functionality. However, there are also techniques with much potential that are not supported by tools yet.

% TODO summarize our own parts


\bibliographystyle{IEEEtran}
\bibliography{AutomatedRefactoring}

% TODO Place appropriately
%\balance

% that's all folks
\end{document}
