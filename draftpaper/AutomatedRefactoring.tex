\documentclass[journal]{IEEEtran}

\usepackage{float}
\usepackage[utf8]{inputenc}
\usepackage[final]{pdfpages}
\usepackage[cmex10]{amsmath}
\usepackage{amssymb}
\usepackage{comment}
\usepackage{setspace}
\usepackage{caption}
\usepackage{subcaption}
\usepackage{acronym}
\usepackage{url}

\usepackage{multirow} 
\usepackage{pbox}
\usepackage{xspace}
\usepackage{balance}

\newcommand{\lk}[1]{\textcolor{orange}{\textbf{[LK] #1}}}
\newcommand{\pf}[1]{\textcolor{green}{\textbf{[PF] #1}}}

\begin{document}
\title{Automated Refactoring}


\author{\IEEEauthorblockN{Lisa Maria Kritzinger\IEEEauthorrefmark{1}, Peter Feichtinger\IEEEauthorrefmark{1}}

\IEEEauthorblockA{\IEEEauthorrefmark{1}k1255353, Email: kritzinger@gmx.net}

\IEEEauthorblockA{\IEEEauthorrefmark{2}kXXXXXXX, Email: shippo@gmx.at}}


% make the title area
\maketitle


\begin{abstract}
	
Text

\end{abstract}

\begin{IEEEkeywords}
Text
\end{IEEEkeywords}

% For peer review papers, you can put extra information on the cover
% page as needed:
% \ifCLASSOPTIONpeerreview
% \begin{center} \bfseries EDICS Category: 3-BBND \end{center}
% \fi
%
% For peerreview papers, this IEEEtran command inserts a page break and
% creates the second title. It will be ignored for other modes.
\IEEEpeerreviewmaketitle


\section{Introduction}

Text \cite{polymorphism}

\section{Background}

Text

\section{Conclusion}

Text

\section*{Acknowledgments}

Text

\bibliographystyle{IEEEtran}
\bibliography{AutomatedRefactoring}

\balance

% that's all folks
\end{document}


