\documentclass[conference,compsoc,a4paper]{IEEEtran}

\usepackage{float}
\usepackage[utf8]{inputenc}
\usepackage[final]{pdfpages}
\usepackage[cmex10]{amsmath}
\usepackage{amssymb}
\usepackage{comment}
\usepackage{setspace}
\usepackage{caption}
\usepackage{subcaption}
\usepackage{acronym}
\usepackage{url}

\usepackage{multirow}
\usepackage{pbox}
\usepackage{xspace}
\usepackage{balance}

\newcommand{\lk}[1]{\textcolor{orange}{\textbf{[LK] #1}}}
\newcommand{\pf}[1]{\textcolor{green}{\textbf{[PF] #1}}}


\begin{document}

\title{Automated Refactoring}
\author{
  \IEEEauthorblockN{Lisa Maria Kritzinger}
  \IEEEauthorblockA{Johannes Kepler University Linz\\
    1255353\\
    Email: kritzinger@gmx.net}
  \and
  \IEEEauthorblockN{Peter Feichtinger}
  \IEEEauthorblockA{Johannes Kepler University Linz\\
    1056451\\
    Email: shippo@gmx.at}
}

% make the title area
\maketitle


\begin{abstract}
Refactoring, the restructuring of a software system without changing its semantics, is essential in software evolution. 
Manual refactoring can be time-consuming and error-prone, so tool support is desirable when making large changes. In 
this article, we will explore a number of publications on automating different refactoring tasks, from just making code 
more compact to introducing objects into a C codebase.

\pf{Should the abstract be longer?}
\end{abstract}

\begin{IEEEkeywords}
Software restructuring, automatic refactoring, tool support, software evolution.
\end{IEEEkeywords}

% For peer review papers, you can put extra information on the cover
% page as needed:
% \ifCLASSOPTIONpeerreview
% \begin{center} \bfseries EDICS Category: 3-BBND \end{center}
% \fi
%
% For peerreview papers, this IEEEtran command inserts a page break and
% creates the second title. It will be ignored for other modes.
\IEEEpeerreviewmaketitle


\section{Introduction}

Text \cite{polymorphism}


\section{Background}

Text


\section{Conclusion}

Text


\section*{Acknowledgments}

Text


\bibliographystyle{IEEEtran}
\bibliography{AutomatedRefactoring}

% TODO Place appropriately
%\balance

% that's all folks
\end{document}
